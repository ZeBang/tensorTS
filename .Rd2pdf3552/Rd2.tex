\documentclass[a4paper]{book}
\usepackage[times,inconsolata,hyper]{Rd}
\usepackage{makeidx}
\usepackage[utf8]{inputenc} % @SET ENCODING@
% \usepackage{graphicx} % @USE GRAPHICX@
\makeindex{}
\begin{document}
\chapter*{}
\begin{center}
{\textbf{\huge Package `timeFA'}}
\par\bigskip{\large \today}
\end{center}
\begin{description}
\raggedright{}
\inputencoding{utf8}
\item[Type]\AsIs{Package}
\item[Title]\AsIs{Factor Models and Autoregressive Models for Time Series}
\item[Version]\AsIs{0.1.0.9000}
\item[Author@R]\AsIs{person(``Zebang'', ``Li'', email = ``zebang.li@rutgers.edu'',
role = c(``aut'', ``cre''))}
\item[Description]\AsIs{More about what it does (maybe more than one line)
Use four spaces when indenting paragraphs within the Description.}
\item[License]\AsIs{GPL (>= 2)}
\item[Encoding]\AsIs{UTF-8}
\item[LazyData]\AsIs{true}
\item[Imports]\AsIs{tensor,
rTensor,
MASS,
abind}
\item[RoxygenNote]\AsIs{7.0.2}
\item[VignetteBuilder]\AsIs{knitr}
\end{description}
\Rdcontents{\R{} topics documented:}
\inputencoding{utf8}
\HeaderA{add}{Add together two numbers.}{add}
%
\begin{Description}\relax
Add together two numbers.
\end{Description}
%
\begin{Usage}
\begin{verbatim}
add(x, y)
\end{verbatim}
\end{Usage}
%
\begin{Arguments}
\begin{ldescription}
\item[\code{x}] A number.

\item[\code{y}] A number.
\end{ldescription}
\end{Arguments}
%
\begin{Value}
The sum of \code{x} and \code{y}.
\end{Value}
%
\begin{Examples}
\begin{ExampleCode}
add(1, 1)
add(10, 1)
\end{ExampleCode}
\end{Examples}
\inputencoding{utf8}
\HeaderA{MAR.SE}{Asymptotic Covariance Matrix of \code{MAR1.otimes}}{MAR.SE}
%
\begin{Description}\relax
Asymptotic covariance Matrix of \code{MAR1.otimes} for given A, B and matrix-valued time series xx, see Theory 3 in paper.
\end{Description}
%
\begin{Usage}
\begin{verbatim}
MAR.SE(xx, B, A, Sigma)
\end{verbatim}
\end{Usage}
%
\begin{Arguments}
\begin{ldescription}
\item[\code{xx}] T * p * q matrix-valued time series

\item[\code{B}] q by q matrix in MAR(1) model

\item[\code{A}] p by p matrix in MAR(1) model

\item[\code{Sig}] covariance matrix cov(vec(E\_t)) in MAR(1) model
\end{ldescription}
\end{Arguments}
%
\begin{Value}
asmptotic covariance matrix
\end{Value}
%
\begin{Examples}
\begin{ExampleCode}
# given T * p * q time series xx
out2=MAR1.LS(xx)
FnormLL=sqrt(sum(out2$LL))
xdim=p;ydim=q
out2Xi=MAR.SE(xx.nm,out2$RR*FnormLL,out2$LL/FnormLL,out2$Sig)
out2SE=sqrt(diag(out2Xi))
SE.A=matrix(out2SE[1:xdim^2],nrow=xdim)
SE.B=t(matrix(out2SE[-(1:xdim^2)],nrow=ydim))
\end{ExampleCode}
\end{Examples}
\inputencoding{utf8}
\HeaderA{MAR1.LS}{Least Squares Iterative Estimation}{MAR1.LS}
%
\begin{Description}\relax
Iterated least squares estimation in the model \eqn{X_t = LL * X_{t-1} * RR + E_t}{}.
\end{Description}
%
\begin{Usage}
\begin{verbatim}
MAR1.LS(xx, niter = 50, tol = 1e-06, print.true = FALSE)
\end{verbatim}
\end{Usage}
%
\begin{Arguments}
\begin{ldescription}
\item[\code{xx}] T * p * q matrix-valued time series

\item[\code{niter}] maximum number of iterations if error stays above \code{tol}

\item[\code{tol}] relative Frobenius norm error tolerance

\item[\code{print.true}] printe LL and RR
\end{ldescription}
\end{Arguments}
%
\begin{Value}
a list containing the following:\begin{description}

\item[\code{LL}] estimator of LL, a p by p matrix
\item[\code{RR}] estimator of RR, a q by q matrix
\item[\code{res}] residual of the MAR(1)
\item[\code{Sig}] covariance matrix cov(vec(E\_t))
\item[\code{dis}] Frobenius norm difference of last update
\item[\code{niter}] number of iterations

\end{description}

\end{Value}
\inputencoding{utf8}
\HeaderA{MAR1.otimes}{MLE under a structured covariance tensor}{MAR1.otimes}
%
\begin{Description}\relax
MAR(1) iterative estimation with Kronecker covariance structure: \eqn{X_t = LL X_{t-1} RR + E_t}{} such that \eqn{Sig = cov(vec(E_t)) = Sigr otimes Sigl}{}.
\end{Description}
%
\begin{Usage}
\begin{verbatim}
MAR1.otimes(
  xx,
  LL.init = NULL,
  Sigl.init = NULL,
  Sigr.init = NULL,
  niter = 50,
  tol = 1e-06,
  print.true = FALSE
)
\end{verbatim}
\end{Usage}
%
\begin{Arguments}
\begin{ldescription}
\item[\code{xx}] T * p * q matrix-valued time series

\item[\code{LL.init}] initial value of LL

\item[\code{Sigl.init}] initial value of Sigl

\item[\code{Sigr.init}] initial value of Sigr

\item[\code{niter}] maximum number of iterations if error stays above \code{tol}

\item[\code{tol}] relative Frobenius norm error tolerance

\item[\code{print.true}] printe LL and RR
\end{ldescription}
\end{Arguments}
%
\begin{Value}
a list containing the following:\begin{description}

\item[\code{LL}] estimator of LL, a p by p matrix
\item[\code{RR}] estimator of RR, a q by q matrix
\item[\code{res}] residual of the MAR(1)
\item[\code{Sigl}] one part of structured covariance matrix Sig=Sigr otimes Sigl
\item[\code{Sigr}] one part of structured covariance matrix Sig=Sigr otimes Sigl
\item[\code{dis}] Frobenius norm difference of last update
\item[\code{niter}] number of iterations

\end{description}

\end{Value}
\inputencoding{utf8}
\HeaderA{MAR1.projection}{Projection Method}{MAR1.projection}
%
\begin{Description}\relax
MAR(1) one step projection estimation in the model \eqn{X_t = LL * X_{t-1} * RR + E_t}{}.
\end{Description}
%
\begin{Usage}
\begin{verbatim}
MAR1.projection(xx)
\end{verbatim}
\end{Usage}
%
\begin{Arguments}
\begin{ldescription}
\item[\code{xx}] T * p * q matrix-valued time series
\end{ldescription}
\end{Arguments}
%
\begin{Value}
a list containing the following:\begin{description}

\item[\code{LL}] estimator of LL, a p by p matrix
\item[\code{RR}] estimator of RR, a q by q matrix
\item[\code{res}] residual of the MAR(1)
\item[\code{Sig}] covariance matrix cov(vec(E\_t))

\end{description}

\end{Value}
\inputencoding{utf8}
\HeaderA{projection}{Kronecker Product Approximation}{projection}
%
\begin{Description}\relax
Kronecker product approximation used in Projection Method of matrix-value time series.
\end{Description}
%
\begin{Usage}
\begin{verbatim}
projection(A, m1, m2, n1, n2)
\end{verbatim}
\end{Usage}
%
\begin{Arguments}
\begin{ldescription}
\item[\code{A}] m by n matrix such that \eqn{m = m1*n1}{} and \eqn{n = m2*n2}{}

\item[\code{m1}] \code{ncol} of A

\item[\code{m2}] \code{ncol} of B

\item[\code{n1}] \code{nrow} of A

\item[\code{n2}] \code{nrow} of B
\end{ldescription}
\end{Arguments}
%
\begin{Value}
a list contaning two estimator (matrix)
\end{Value}
%
\begin{SeeAlso}\relax
\code{\LinkA{MAR1.projection}{MAR1.projection}}
\end{SeeAlso}
%
\begin{Examples}
\begin{ExampleCode}
A <- matrix(runif(6),ncol=2),
projection(A,3,3,2,2)
\end{ExampleCode}
\end{Examples}
\inputencoding{utf8}
\HeaderA{rearrange}{Rearrangement Operator}{rearrange}
%
\begin{Description}\relax
Rearrangement Operator used for projection method.
\end{Description}
%
\begin{Usage}
\begin{verbatim}
rearrange(A, m1, m2, n1, n2)
\end{verbatim}
\end{Usage}
%
\begin{Arguments}
\begin{ldescription}
\item[\code{A}] m by n matrix such that \eqn{m = m1*n1}{} and \eqn{n = m2*n2}{}

\item[\code{m1}] \code{ncol} of A

\item[\code{m2}] \code{ncol} of B

\item[\code{n1}] \code{nrow} of A

\item[\code{n2}] \code{nrow} of B
\end{ldescription}
\end{Arguments}
%
\begin{Value}
rearengement matrix
\#'@seealso \code{\LinkA{MAR1.projection}{MAR1.projection}}
\end{Value}
%
\begin{Examples}
\begin{ExampleCode}
A <- matrix(runif(6),ncol=2),
B <- matrix(runif(6),ncol=2),
M <- kronecker(B,A)
rearrange(M,3,3,2,2) == t(as.vector(A)) %*% as.vector(B)
'TRUE'
\end{ExampleCode}
\end{Examples}
\inputencoding{utf8}
\HeaderA{timeFA}{timeFA: factor and autoregressive models for high dimensional time series.}{timeFA}
%
\begin{Description}\relax
The timefa package provides two categories of models:
factor models and autoregressive models.
\end{Description}
%
\begin{Section}{factor model functions}

The functions ...
\end{Section}
%
\begin{Section}{autoregressive model functions}

The functions ...
\end{Section}
\inputencoding{utf8}
\HeaderA{var1}{Stacked vector AR(1) Model}{var1}
%
\begin{Description}\relax
vector AR(1) Model.
\end{Description}
%
\begin{Usage}
\begin{verbatim}
var1(xx)
\end{verbatim}
\end{Usage}
%
\begin{Arguments}
\begin{ldescription}
\item[\code{xx}] T * p * q matrix-valued time series
\end{ldescription}
\end{Arguments}
%
\begin{Value}
a list containing the following:\begin{description}

\item[\code{coef}] coeficient of the fitted VAR(1) model
\item[\code{res}] residual of the VAR(1) model

\end{description}

\end{Value}
%
\begin{Examples}
\begin{ExampleCode}
out.var1=var1(xx)
sum(out.var1$res**2)
\end{ExampleCode}
\end{Examples}
\printindex{}
\end{document}
